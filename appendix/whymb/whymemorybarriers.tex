% appendix/whymb/whymemorybarriers.tex
% mainfile: ../../perfbook.tex
% SPDX-License-Identifier: CC-BY-SA-3.0

\QuickQuizChapter{chp:app:whymb:Why Memory Barriers?}{Why Memory Barriers?}{qqzwhymb}
%
\Epigraph{Order!
	  Order in the court!}
	 {Unknown}

So what possessed CPU designers to cause them to inflict \IXBpl{memory barrier}
on poor unsuspecting SMP software designers?

In short, because reordering memory references allows much better performance,
courtesy of the finite speed of light and the non-zero size of atoms
noted in \cref{sec:cpu:Overheads}, and particularly in the
hardware-performance question posed by \QuickQuizRef{\QspeedOfLightAtoms}.
Therefore, memory barriers are needed to force ordering in things like
synchronization primitives whose correct operation depends on ordered
memory references.

Getting a more detailed answer to this question requires a good understanding
of how CPU caches work, and especially what is required to make
caches really work well.
The following sections:
\begin{enumerate}
\item	Present the structure of a cache,
\item	Describe how cache-coherency protocols ensure that CPUs agree
	on the value of each location in memory, and, finally,
\item	Outline how store buffers and invalidate queues help
	caches and cache-coherency protocols achieve high performance.
\end{enumerate}
We will see that memory barriers are a necessary evil that is required
to enable good performance and scalability, an evil that stems from
the fact that CPUs are orders of magnitude faster than are both the
interconnects between them and the memory they are attempting to access.

\section{Cache Structure}
\label{sec:app:whymb:Cache Structure}

Modern CPUs are much faster than are modern memory systems.
A 2006 CPU might be capable of executing ten instructions per nanosecond,
but will require many tens of nanoseconds to fetch a data item from
main memory.
This disparity in speed---more than two orders of magnitude---has
resulted in the multi-megabyte caches found on modern CPUs.
These caches are associated with the CPUs as shown in
\cref{fig:app:whymb:Modern Computer System Cache Structure},
and can typically be accessed in a few cycles.\footnote{
	It is standard practice to use multiple levels of cache,
	with a small level-one cache close to the CPU with
	single-cycle access time, and a larger level-two cache
	with a longer access time, perhaps roughly ten clock cycles.
	Higher-performance CPUs often have three or even four levels
	of cache.}

\begin{figure}
\centering
\resizebox{3in}{!}{\includegraphics{appendix/whymb/cacheSC}}
\caption{Modern Computer System Cache Structure}
\label{fig:app:whymb:Modern Computer System Cache Structure}
\end{figure}

Data flows among the CPUs' caches and memory in fixed-length blocks
called ``\IXpl{cache line}'', which are normally a power of two in size,
ranging from 16 to 256 bytes.
When a given data item is first accessed by a given CPU, it will
be absent from that CPU's cache, meaning that a ``cache miss''
(or, more specifically, a ``startup'' or ``warmup'' cache miss)
has occurred.
The cache miss means that the CPU will
have to wait (or be ``stalled'') for hundreds of cycles while the
item is fetched from memory.
However, the item will be loaded into that CPU's cache, so that
subsequent accesses will find it in the cache and therefore run
at full speed.

After some time, the CPU's cache will fill, and subsequent
misses will likely need to eject an item from the cache in order
to make room for the newly fetched item.
Such a cache miss is termed a ``\IXalth{capacity miss}
{capacity}{cache miss}'', because it is caused
by the cache's limited capacity.
However, most caches can be forced to eject an old item to make room
for a new item even when they are not yet full.
This is due to the fact that large caches are implemented as hardware
hash tables with fixed-size hash buckets (or ``sets'', as CPU designers
call them) and no chaining, as shown in
\cref{fig:app:whymb:CPU Cache Structure}.

This cache has sixteen ``sets'' and two ``ways'' for a total of 32
``lines'', each entry containing a single 256-byte ``cache line'',
which is a 256-byte-aligned block of memory.
This cache line size is a little on the large size, but makes the hexadecimal
arithmetic much simpler.
In hardware parlance, this is a two-way set-associative cache, and
is analogous to a software hash table with
sixteen buckets, where each bucket's hash chain is limited to
at most two elements.
The size (32 cache lines in this case) and the
\IXalt{associativity}{cache associativity} (two in
this case) are collectively called the cache's
``\IXalt{geometry}{cache geometry}''.
Since this cache is implemented in hardware, the hash function is
extremely simple:
Extract four bits from the memory address.

\begin{figure}
\centering
\small
\begin{picture}(170,170)(0,0)

	% Addresses

	\put(0,0){\makebox(20,10){\tt 0xF}}
	\put(0,10){\makebox(20,10){\tt 0xE}}
	\put(0,20){\makebox(20,10){\tt 0xD}}
	\put(0,30){\makebox(20,10){\tt 0xC}}
	\put(0,40){\makebox(20,10){\tt 0xB}}
	\put(0,50){\makebox(20,10){\tt 0xA}}
	\put(0,60){\makebox(20,10){\tt 0x9}}
	\put(0,70){\makebox(20,10){\tt 0x8}}
	\put(0,80){\makebox(20,10){\tt 0x7}}
	\put(0,90){\makebox(20,10){\tt 0x6}}
	\put(0,100){\makebox(20,10){\tt 0x5}}
	\put(0,110){\makebox(20,10){\tt 0x4}}
	\put(0,120){\makebox(20,10){\tt 0x3}}
	\put(0,130){\makebox(20,10){\tt 0x2}}
	\put(0,140){\makebox(20,10){\tt 0x1}}
	\put(0,150){\makebox(20,10){\tt 0x0}}

	% Way 0

	\put(20,163){\makebox(80,10){Way 0}}
	\put(20,0){\framebox(80,10){\tt }}
	\put(20,10){\framebox(80,10){\tt 0x12345E00}}
	\put(20,20){\framebox(80,10){\tt 0x12345D00}}
	\put(20,30){\framebox(80,10){\tt 0x12345C00}}
	\put(20,40){\framebox(80,10){\tt 0x12345B00}}
	\put(20,50){\framebox(80,10){\tt 0x12345A00}}
	\put(20,60){\framebox(80,10){\tt 0x12345900}}
	\put(20,70){\framebox(80,10){\tt 0x12345800}}
	\put(20,80){\framebox(80,10){\tt 0x12345700}}
	\put(20,90){\framebox(80,10){\tt 0x12345600}}
	\put(20,100){\framebox(80,10){\tt 0x12345500}}
	\put(20,110){\framebox(80,10){\tt 0x12345400}}
	\put(20,120){\framebox(80,10){\tt 0x12345300}}
	\put(20,130){\framebox(80,10){\tt 0x12345200}}
	\put(20,140){\framebox(80,10){\tt 0x12345100}}
	\put(20,150){\framebox(80,10){\tt 0x12345000}}

	% Way 1

	\put(100,163){\makebox(80,10){Way 1}}
	\put(100,0){\framebox(80,10){\tt }}
	\put(100,10){\framebox(80,10){\tt 0x43210E00}}
	\put(100,20){\framebox(80,10){\tt }}
	\put(100,30){\framebox(80,10){\tt }}
	\put(100,40){\framebox(80,10){\tt }}
	\put(100,50){\framebox(80,10){\tt }}
	\put(100,60){\framebox(80,10){\tt }}
	\put(100,70){\framebox(80,10){\tt }}
	\put(100,80){\framebox(80,10){\tt }}
	\put(100,90){\framebox(80,10){\tt }}
	\put(100,100){\framebox(80,10){\tt }}
	\put(100,110){\framebox(80,10){\tt }}
	\put(100,120){\framebox(80,10){\tt }}
	\put(100,130){\framebox(80,10){\tt }}
	\put(100,140){\framebox(80,10){\tt }}
	\put(100,150){\framebox(80,10){\tt }}

\end{picture}
\caption{CPU Cache Structure}
\label{fig:app:whymb:CPU Cache Structure}
\end{figure}

In \cref{fig:app:whymb:CPU Cache Structure},
each box corresponds to a cache entry, which
can contain a 256-byte cache line.
However, a cache entry can be empty, as indicated by the empty boxes
in the figure.
The rest of the boxes are flagged with the memory address of the cache line
that they contain.
Since the cache lines must be 256-byte aligned, the low eight bits of
each address are
zero, and the choice of hardware hash function means that the next-higher
four bits match the hash line number.

The situation depicted in the figure might arise if the program's code
were located at address 0x43210E00 through 0x43210EFF, and this program
accessed data sequentially from 0x12345000 through 0x12345EFF\@.
Suppose that the program were now to access location 0x12345F00.
This location hashes to line 0xF, and both ways of this line are
empty, so the corresponding 256-byte line can be accommodated.
If the program were to access location 0x1233000, which hashes to line
0x0, the corresponding 256-byte cache line can be accommodated in
way 1.
However, if the program were to access location 0x1233E00, which hashes
to line 0xE, one of the existing lines must be ejected from the cache
to make room for the new cache line.
If this ejected line were accessed later, a cache miss would result.
Such a cache miss is termed an ``\IXalth{associativity miss}
{associativity}{cache miss}''.

Thus far, we have been considering only cases where a CPU reads
a data item.
What happens when it does a write?
Because it is important that all CPUs agree on the value of a given
data item, before a given CPU writes to that data item, it must first
cause it to be removed, or ``invalidated'', from other CPUs' caches.
Once this \IX{invalidation} has completed, the CPU may safely modify the
data item.
If the data item was present in this CPU's cache, but was read-only,
this process is termed a ``\IXalth{write miss}{write}{cache miss}''.
Once a given CPU has completed invalidating a given data item from other
CPUs' caches, that CPU may repeatedly write (and read) that data item.

Later, if one of the other CPUs attempts to access the data item, it
will incur a cache miss, this time because the first CPU invalidated
the item in order to write to it.
This type of cache miss is termed a ``\IXalth{communication miss}
{communication}{cache miss}'', since it
is usually due to several CPUs using the data items to communicate
(for example, a lock is a data item that is used to communicate among
CPUs using a mutual-exclusion algorithm).

Clearly, much care must be taken to ensure that all CPUs maintain
a coherent view of the data.
With all this fetching, invalidating, and writing, it is easy to
imagine data being lost or (perhaps worse) different CPUs having
conflicting values for the same data item in their respective
caches.
These problems are prevented by ``cache-coherency protocols'',
described in the next section.

\section{Cache-Coherence Protocols}
\label{sec:app:whymb:Cache-Coherence Protocols}

\IXpl{Cache-coherence protocol} manage cache-line states so as to prevent
inconsistent or lost data.
These protocols can be quite complex, with many tens
of states,\footnote{
	See Culler et al.~\cite{DavidECuller1999} pages 670 and 671
	for the nine-state and 26-state diagrams for SGI Origin2000
	and Sequent (now IBM) NUMA-Q, respectively.
	Both diagrams are significantly simpler than real life.}
but for our purposes we need only concern ourselves with the
four-state \IXaltr{MESI cache-coherence protocol}{MESI protocol}.

\subsection{MESI States}
\label{sec:app:whymb:MESI States}

MESI stands for ``modified'', ``exclusive'', ``shared'', and ``invalid'',
the four states a given cache line can take on using this
protocol.
Caches using this protocol therefore maintain a two-bit state ``tag'' on each
cache line in addition to that line's physical address and data.
% cite Schimmel's book on virtual caches.

A line in the ``modified'' state has been subject to a recent memory store
from the corresponding CPU, and the corresponding memory is guaranteed
not to appear in any other CPU's cache.
Cache lines in the ``modified'' state can thus be said to be ``owned''
by the CPU\@.
Because this cache holds the only up-to-date copy of the data, this
cache is ultimately responsible for either writing it back to memory
or handing it off to some other cache, and must do so before reusing
this line to hold other data.

The ``exclusive'' state is very similar to the ``modified'' state,
the single exception being that the cache line has not yet been
modified by the corresponding CPU, which in turn means that the
copy of the cache line's data that resides in memory is up-to-date.
However, since the CPU can store to this line at any time, without
consulting other CPUs, a line in the ``exclusive'' state can still
be said to be owned by the corresponding CPU\@.
That said, because the corresponding value in memory is up to date,
this cache can discard this data without writing it back to memory
or handing it off to some other CPU\@.

A line in the ``shared'' state might be replicated in at least
one other CPU's cache, so that this CPU is not permitted to store
to the line without first consulting with other CPUs.
As with the ``exclusive'' state, because the corresponding value
in memory is up to date,
this cache can discard this data without writing it back to memory
or handing it off to some other CPU\@.

A line in the ``invalid'' state is empty, in other words, it holds
no data.
When new data enters the cache, it is placed into a
cache line that was in the ``invalid'' state if possible.
This approach is preferred because replacing a line in any other
state could result in an expensive cache miss should the replaced
line be referenced in the future.

Since all CPUs must maintain a coherent view of the data carried in
the cache lines, the cache-coherence protocol provides messages
that coordinate the movement of cache lines through the system.

\subsection{MESI Protocol Messages}
\label{sec:app:whymb:MESI Protocol Messages}

Many of the transitions described in the previous section require
communication among the CPUs.
If the CPUs are on a single shared bus, the following messages suffice:
\begin{description}[style=nextline]
\item	[Read:]
	The ``read'' message contains the physical address of the cache line
	to be read.
\item	[Read Response:]
	The ``read response'' message contains the data requested by an
	earlier ``read'' message.
	This ``read response'' message might be supplied either by
	memory or by one of the other caches.
	For example, if one of the caches has the desired data in
	``modified'' state, that cache must supply the ``read response''
	message.
\item	[Invalidate:]
	The ``invalidate'' message contains the physical address of the
	cache line to be invalidated.
	All other caches must remove the corresponding data from their
	caches and respond.
\item	[Invalidate Acknowledge:]
	A CPU receiving an ``invalidate'' message must respond with an
	``invalidate acknowledge'' message after removing the specified
	data from its cache.
\item	[Read Invalidate:]
	The ``read invalidate'' message contains the physical address
	of the cache line to be read, while at the same time directing
	other caches to remove the data.
	Hence, it is a combination of a ``read'' and an ``invalidate'',
	as indicated by its name.
	A ``read invalidate'' message requires both a ``read response''
	and a set of ``invalidate acknowledge'' messages in reply.
\item	[Writeback:]
	The ``writeback'' message contains both the address and the
	data to be written back to memory (and perhaps ``snooped''
	into other CPUs' caches along the way).
	This message permits caches to eject lines in the ``modified''
	state as needed to make room for other data.
\end{description}

\QuickQuiz{
	Where does a writeback message originate from and where does
	it go to?
}\QuickQuizAnswer{
	The writeback message originates from a given CPU, or in some
	designs from a given level of a given CPU's cache---or even
	from a cache that might be shared among several CPUs.
	The key point is that a given cache does not have room for
	a given data item, so some other piece of data must be ejected
	from the cache to make room.
	If there is some other piece of data that is duplicated in some
	other cache or in memory, then that piece of data may be simply
	discarded, with no writeback message required.

	On the other hand, if every piece of data that might be ejected
	has been modified so that the only up-to-date copy is in this
	cache, then one of those data items must be copied somewhere
	else.
	This copy operation is undertaken using a ``writeback message''.

	The destination of the writeback message has to be something
	that is able to store the new value.
	This might be main memory, but it also might be some other cache.
	If it is a cache, it is normally a higher-level cache for the
	same CPU, for example, a level-1 cache might write back to a
	level-2 cache.
	However, some hardware designs permit cross-CPU writebacks,
	so that CPU~0's cache might send a writeback message to CPU~1.
	This would normally be done if CPU~1 had somehow indicated
	an interest in the data, for example, by having recently
	issued a read request.

	In short, a writeback message is sent from some part of the
	system that is short of space, and is received by some other
	part of the system that can accommodate the data.
}\QuickQuizEnd

Interestingly enough, a shared-memory multiprocessor system really
is a message-passing computer under the covers.
This means that clusters of SMP machines that use distributed shared memory
are using message passing to implement shared memory at two different
levels of the system architecture.

\QuickQuizSeries{%
\QuickQuizB{
	What happens if two CPUs attempt to invalidate the
	same cache line concurrently?
}\QuickQuizAnswerB{
	One of the CPUs gains access to the shared bus first,
	and that CPU ``wins''.
	The other CPU must invalidate its copy of the cache line and
	transmit an ``invalidate acknowledge'' message to the other CPU\@.

	Of course, the losing CPU can be expected to immediately issue a
	``read invalidate'' transaction, so the winning CPU's victory will
	be quite ephemeral.
}\QuickQuizEndB
%
\QuickQuizM{
	When an ``invalidate'' message appears in a large multiprocessor,
	every CPU must give an ``invalidate acknowledge'' response.
	Wouldn't the resulting ``storm'' of ``invalidate acknowledge''
	responses totally saturate the system bus?
}\QuickQuizAnswerM{
	It might, if large-scale multiprocessors were in fact implemented
	that way.
	Larger multiprocessors, particularly NUMA machines,
	tend to use so-called ``directory-based'' cache-coherence
	protocols to avoid this and other problems.
}\QuickQuizEndM
%
\QuickQuizE{
	If SMP machines are really using message passing
	anyway, why bother with SMP at all?
}\QuickQuizAnswerE{
	There has been quite a bit of controversy on this topic over
	the past few decades.
	One answer is that the cache-coherence
	protocols are quite simple, and therefore can be implemented
	directly in hardware, gaining bandwidths and latencies
	unattainable by software message passing.
	Another answer is that
	the real truth is to be found in economics due to the relative
	prices of large SMP machines and that of clusters of smaller
	SMP machines.
	A third answer is that the SMP programming model is easier to
	use than that of distributed systems, but a rebuttal might note
	the appearance of HPC clusters and MPI\@.
	And so the argument continues.
}\QuickQuizEndE
}

\subsection{MESI State Diagram}
\label{sec:app:whymb:MESI State Diagram}

A given cache line's state changes
as protocol messages are sent and received, as
shown in \cref{fig:app:whymb:MESI Cache-Coherency State Diagram}.

\begin{figure}
\centering
% \resizebox{3in}{!}{\includegraphics{appendix/whymb/MESI}}
\includegraphics{appendix/whymb/MESI}
\caption{MESI Cache-Coherency State Diagram}
\label{fig:app:whymb:MESI Cache-Coherency State Diagram}
\end{figure}

The transition arcs in this figure are as follows:
\begin{description}[style=nextline]
\item	[Transition (a):]
	A cache line is written back to memory, but the CPU retains
	it in its cache and further retains the right to modify it.
	This transition requires a ``writeback'' message.
\item	[Transition (b):]
	The CPU writes to the cache line that it already had exclusive
	access to.
	This transition does not require any messages to be sent or
	received.
\item	[Transition (c):]
	The CPU receives a ``read invalidate'' message for a cache line
	that it has modified.
	The CPU must invalidate its local copy, then respond with both a
	``read response'' and an ``invalidate acknowledge'' message,
	both sending the data to the requesting CPU and indicating
	that it no longer has a local copy.
\item	[Transition (d):]
	The CPU does an \IX{atomic read-modify-write operation} on a data item
	that was not present in its cache.
	It transmits a ``read invalidate'', receiving the data via
	a ``read response''.
	The CPU can complete the transition once it has also received a
	full set of ``invalidate acknowledge'' responses.
\item	[Transition (e):]
	The CPU does an atomic read-modify-write operation on a data item
	that was previously read-only in its cache.
	It must transmit ``invalidate'' messages, and must wait for a
	full set of ``invalidate acknowledge'' responses before completing
	the transition.
\item	[Transition (f):]
	Some other CPU reads the cache line, and it is supplied from
	this CPU's cache, which retains a read-only copy, possibly also
	writing it back to memory.
	This transition is initiated by the reception of a ``read''
	message, and this CPU responds with a ``read response'' message
	containing the requested data.
\item	[Transition (g):]
	Some other CPU reads a data item in this cache line,
	and it is supplied either from this CPU's cache or from memory.
	In either case, this CPU retains a read-only copy.
	This transition is initiated by the reception of a ``read''
	message, and this CPU responds with a ``read response'' message
	containing the requested data.
\item	[Transition (h):]
	This CPU realizes that it will soon need to write to some data
	item in this cache line, and thus transmits an ``invalidate'' message.
	The CPU cannot complete the transition until it receives a full
	set of ``invalidate acknowledge'' responses, indicating that
	no other CPU has this cacheline in its cache.
	In other words, this CPU is the only CPU caching it.
\item	[Transition (i):]
	Some other CPU does an atomic read-modify-write operation on
	a data item in a cache line held only in this CPU's cache,
	so this CPU invalidates it from its cache.
	This transition is initiated by the reception of a ``read invalidate''
	message, and this CPU responds with both a ``read response''
	and an ``invalidate acknowledge'' message.
\item	[Transition (j):]
	This CPU does a store to a data item in a cache line that was not
	in its cache, and thus transmits a ``read invalidate'' message.
	The CPU cannot complete the transition until it receives the
	``read response'' and a full set of ``invalidate acknowledge''
	messages.
	The cache line will presumably transition to ``modified'' state via
	transition (b) as soon as the actual store completes.
\item	[Transition (k):]
	This CPU loads a data item in a cache line that was not
	in its cache.
	The CPU transmits a ``read'' message, and completes the
	transition upon receiving the corresponding ``read response''.
\item	[Transition (l):]
	Some other CPU does a store to
	a data item in this cache line, but holds this cache line in read-only
	state due to its being held in other CPUs' caches (such as the
	current CPU's cache).
	This transition is initiated by the reception of an ``invalidate''
	message, and this CPU responds with
	an ``invalidate acknowledge'' message.
\end{description}

\QuickQuiz{
	How does the hardware handle the delayed transitions
	described above?
}\QuickQuizAnswer{
	Usually by adding additional states, though these additional
	states need not be actually stored with the cache line, due to
	the fact that only a few lines at a time will be transitioning.
	The need to delay transitions is but one issue that results in
	real-world cache coherence protocols being much more complex than
	the over-simplified MESI protocol described in this appendix.
	Hennessy and Patterson's classic introduction to computer
	architecture~\cite{Hennessy95a} covers many of these issues.
}\QuickQuizEnd

\subsection{MESI Protocol Example}
\label{sec:app:whymb:MESI Protocol Example}

Let's now look at this from the perspective of a cache line's worth
of data, initially residing in memory at address~0,
as it travels through the various single-line \IXhpl{direct-mapped}{cache}
in a four-CPU system.
\Cref{tab:app:whymb:Cache Coherence Example}
shows this flow of data, with the first column showing the steps
each operation takes, the second the CPU performing the operation,
the third the operation being performed, the next four the state
of each CPU's cache line (memory address followed by MESI state),
and the final two columns whether the corresponding memory contents
are up to date (``V'') or not (``I'').

\begin{table*}
\small
\centering
\ebresizewidth{
\renewcommand*{\arraystretch}{1.2}
\rowcolors{6}{}{lightgray}
\begin{tabular}{lclcccccc}
	\toprule
	& & & \multicolumn{4}{c}{CPU Cache} & \multicolumn{2}{c}{Memory} \\
	\cmidrule(lr){4-7} \cmidrule(l){8-9}
	Step~\# & CPU \# & Operation & 0 & 1 & 2 & 3 & 0 & 8 \\
	\cmidrule(r){1-3} \cmidrule(lr){4-7} \cmidrule(l){8-9}
%	Step CPU Operation	------------- CPU -------------   - Memory -
%				   0	   1	   2	   3	    0   8
	0   &   & Initial State	& $-$/I & $-$/I & $-$/I & $-$/I   & V & V \\
	1   & 0 & Load 0 (cache value miss)
				& 0/S   & $-$/I & $-$/I & $-$/I   & V & V \\
	2   & 3 & Load 0 (cache value miss)
				& 0/S   & $-$/I & $-$/I & 0/S     & V & V \\
	3.1 & 0 & Load 8 (collision, invalidation)
				& $-$/I & $-$/I & $-$/I & 0/S     & V & V \\
	3.2 & 0 & Load 8 (cache value miss)
				& 8/S   & $-$/I & $-$/I & 0/S     & V & V \\
	4   & 3 & Load 0 (cache value hit)
				& 8/S   & $-$/I & $-$/I & 0/I     & V & V \\
	5.1 & 2 & Atomic Inc 0 (cache value miss)
				& 8/S   & $-$/I & 0/S   & 0/S     & V & V \\
	5.2 & 2 & Atomic Inc 0 (cache permission miss)
				& 8/S   & $-$/I & 0/E   & $-$/I   & V & V \\
	5.3 & 2 & Atomic Inc 0	& 8/S   & $-$/I & 0/M   & $-$/I   & I & V \\
	6.1 & 1 & Store 8 (cache value miss)
				& 8/S   & 8/S   & 0/M   & $-$/I   & I & V \\
	6.2 & 1 & Store 8 (cache permission miss)
				& $-$/I & 8/E   & 0/M   & $-$/I   & I & V \\
	6.3 & 1 & Store 8 	& $-$/I & 8/M   & 0/M   & $-$/I   & I & I \\
	7   & 2 & Atomic Inc 0	& $-$/I & 8/M   & 0/M   & $-$/I   & I & I \\
	8.1 & 1 & Load 0 (collision, flush)
				& $-$/I & 8/S   & 0/M   & $-$/I   & I & V \\
	8.2 & 1 & Load 0 (collision, invalidation)
				& $-$/I & $-$/I & 0/M   & $-$/I   & I & V \\
	8.3 & 1 & Load 0 (cache value miss)
				& $-$/I & 0/S   & 0/S   & $-$/I   & I & V \\
	\bottomrule
\end{tabular}
}
\caption{Cache Coherence Example}
\label{tab:app:whymb:Cache Coherence Example}
\end{table*}

Initially, the CPU cache lines in which the data would reside are
in the ``invalid'' state, and the data is valid in memory, as shown
in step~0.
In step~1, CPU~0 loads the data at address~0, incuring a cache miss,
loading the value into its cache, and transitioning that cache line to
the ``shared'' state.
The contents of memory at address~0 remain valid.
In step~2, CPU~3 also loads the data at address~0, also incurring a
cache miss, loading the data into its cache, and transitioning that cache
line to the ``shared'' state.
The cache line corresponding to address~0 is now in ``shared'' state in
both CPUs' caches, and the contents of memory at address~0 remain valid.

In step~3.1, CPU~0 loads the data at address~8.
However, its cache still contains the value from address~0, and its
single-line direct-mapped cache cannot accommodate two values at once.
The CPU therefore forces the data at address~0 out of its cache via
an invalidation, and then step~3.2 then replaces it with the data at
address~8.
The cache line corresponding to address~8 is now in ``shared'' state in
CPU~0's cache, and the contents of memory at both addresses remain valid.
In step~4, CPU~3 repeats its load of the data at address~0, and because
this data remains in CPU~3's cache, this CPU enjoys a cache hit.
As a result, cache and memory state remains unchanged.

In step~5.1, CPU~2 commences atomically incrementing the data at
address~0, incurring a cache miss, loading the data into its cache
(obtaining the data from either CPU~3 or from memory), and transitioning
that cache line to the ``shared'' state.
In step~5.2, CPU~2 continues its atomic increment, which it could not
previously complete while its cache line was in ``shared'' state.
So it causes CPU~3 to invalidate this address from its cache, transitioning
CPU~2's cache line to the ``exclusive'' state.
This permits step~5.3 to complete CPU~2's atomic increment, transitioning
its cache line to the ``modified'' state.
Note that the data residing in memory at address~0 is now out of date.

In step~6.1, CPU~1 commences a single-byte store to address~8, which
must leave the other seven bytes unchanged.
CPU~1 must therefore load this data, incurring a cache miss,
loading the data into its cache (obtaining the data from either CPU~0
or from memory), and transitioning that cache line to the shared state.
In step~6.2, CPU~1 continues its store, which it could not previously
complete while its cache line was in ``shared'' state.
So it causes CPU~0 to invalidate this address from its cache, transitioning
CPU~1's cache line to the ``exclusive'' state.
This permits step~6.3 to complete CPU~1's store, transitioning its cache
line to the ``exclusive'' state.
Note that the data residing in memory at both address~0 and address~8
is now out of date.

In step~7, CPU~2 does another atomic increment of the data at address~0.
However, because its cache holds that data in the ``modified'' state,
CPU~2 can immediately execute that atomic increment without any change
to cache or memory state.

In step~8.1, CPU~1 loads loads the data at address~0, incuring a cache
miss.
Because the data from address~8 is modified in its cache, it must
first flush that data to memory, transitioning its cache line to
``shared'' state and causing the memory at address~8 to once again
become valid.
Step~8.1 can then invalidate CPU~1's cache line, so that
step~8.2 can obtain the value from CPU~2, keeping in mind that
the data in memory at address~0 is invalid.
Both CPUs~1 and~2 transition their cache state to ``shared'',
and the contents of memory at address~0 remain invalid.

Note that we end with data in some of the CPU's caches.

\QuickQuiz{
	What sequence of operations would put the CPUs' caches
	all back into the ``invalid'' state?
}\QuickQuizAnswer{
	There is no such sequence, at least in absence of special
	``flush my cache'' instructions in the CPU's instruction set.
	Most CPUs do have such instructions.
}\QuickQuizEnd

\section{Stores Result in Unnecessary Stalls}
\label{sec:app:whymb:Stores Result in Unnecessary Stalls}

Although the cache structure shown in
\cref{fig:app:whymb:Modern Computer System Cache Structure}
provides good performance for repeated reads and writes from a given CPU
to a given item of data, its performance for the first write to
a given cache line is quite poor.
To see this, consider
\cref{fig:app:whymb:Writes See Unnecessary Stalls},
which shows a timeline of a write by CPU~0 to a cacheline held in
CPU~1's cache.
Since CPU~0 must wait for the cache line to arrive before it can
write to it, CPU~0 must stall for an extended period of time.\footnote{
	The time required to transfer a cache line from one CPU's cache
	to another's is typically a few orders of magnitude more than
	that required to execute a simple register-to-register instruction.}

\begin{figure}
\centering
% \resizebox{3in}{!}{\includegraphics{appendix/whymb/cacheSCwrite}}
\includegraphics{appendix/whymb/cacheSCwrite}
\caption{Writes See Unnecessary Stalls}
\label{fig:app:whymb:Writes See Unnecessary Stalls}
\end{figure}

But there is no real reason to force CPU~0 to stall for so long---after
all, regardless of what data happens to be in the cache line that CPU~1
sends it, CPU~0 is going to unconditionally overwrite it.

\subsection{Store Buffers}
\label{sec:app:whymb:Store Buffers}

One way to prevent this unnecessary stalling of writes is to add
``store buffers'' between each CPU and its cache, as shown in
\cref{fig:app:whymb:Caches With Store Buffers}.
With the addition of these store buffers, CPU~0 can simply record
its write in its store buffer and continue executing.
When the cache line does finally make its way from CPU~1 to CPU~0,
the data will be moved from the store buffer to the cache line.

\QuickQuiz{
	But then why do uniprocessors also have store buffers?
}\QuickQuizAnswer{
	Because the purpose of store buffers is not just to hide
	acknowledgement latencies in multiprocessor cache-coherence protocols,
	but to hide memory latencies in general.
	Because memory is much slower than is cache on uniprocessors,
	store buffers on uniprocessors can help to hide write-miss
	memory latencies.
}\QuickQuizEnd

Please note that the store buffer does not necessarily operate on
full cache lines.
The reason for this is that a given store-buffer entry need only contain
the value stored, not the other data contained in the corresponding
cache line.
Which is a good thing, because the CPU doing the store has no idea
what that other data might be!
But once the corresponding cache line arrives, any values from the
store buffer that update that cache line can be merged into it,
and the corresponding entries can then be removed from the store buffer.
Any other data in that cache line is of course left intact.

\QuickQuiz{
	So store-buffer entries are variable length?
	Isn't that difficult to implement in hardware?
}\QuickQuizAnswer{
	Here are two ways for hardware to easily handle variable-length
	stores.

	First, each store-buffer entry could be a single byte wide.
	Then an 64-bit store would consume eight store-buffer entries.
	This approach is simple and flexible, but one disadvantage is
	that each entry would need to replicate much of the address that
	was stored to.

	Second, each store-buffer entry could be double the size of a
	cache line, with half of the bits containing the values stored,
	and the other half indicating which bits had been stored to.
	So, assuming a 32-bit cache line, a single-byte store of 0x5a
	to the low-order byte of a given cache line would result in
	\co{0xXXXXXX5a} for the first half and \co{0x000000ff} for the
	second half, where the values labeled \co{X} are arbitrary
	because they would be ignored.
	This approach allows multiple consecutive stores corresponding to
	a given cache line to be merged into a single store-buffer entry,
	but is space-inefficient for random stores of single bytes.

	Much more complex and efficient schemes are of course used
	by actual hardware designers.
}\QuickQuizEnd

\begin{figure}
\centering
\resizebox{3in}{!}{\includegraphics{appendix/whymb/cacheSB}}
\caption{Caches With Store Buffers}
\label{fig:app:whymb:Caches With Store Buffers}
\end{figure}

These store buffers are local to a given CPU or, on systems with
hardware multithreading, local to a given core.
Either way, a given CPU is permitted to access only the store buffer
assigned to it.
For example, in
\cref{fig:app:whymb:Caches With Store Buffers}, CPU~0 cannot
access CPU~1's store buffer and vice versa.
This restriction simplifies the hardware by separating concerns:
The store buffer improves performance for consecutive writes, while
the responsibility for communicating among CPUs (or cores, as the
case may be) is fully shouldered by the cache-coherence protocol.
However, even given this restriction, there are complications that must
be addressed, which are covered in the next two sections.

\subsection{Store Forwarding}
\label{sec:app:whymb:Store Forwarding}

To see the first complication, a violation of self-consistency,
consider the following code with variables \qco{a} and \qco{b} both initially
zero, and with the cache line containing variable \qco{a} initially
owned by CPU~1 and that containing \qco{b} initially owned by CPU~0:

\begin{VerbatimN}[fontsize=\footnotesize,samepage=true]
a = 1;
b = a + 1;
assert(b == 2);
\end{VerbatimN}

One would not expect the assertion to fail.
However, if one were foolish enough to use the very simple architecture
shown in
\cref{fig:app:whymb:Caches With Store Buffers},
one would be surprised.
Such a system could potentially see the following sequence of events:
\begin{sequence}
\item	CPU~0 starts executing the \co{a = 1}.
\item	CPU~0 looks \qco{a} up in the cache, and finds that it is missing.
\item	CPU~0 therefore sends a ``read invalidate'' message in order to
	get exclusive ownership of the cache line containing \qco{a}.
\item	CPU~0 records the store to \qco{a} in its store buffer.
\item	CPU~1 receives the ``read invalidate'' message, and responds
	by transmitting the cache line and removing that cacheline from
	its cache.
\item	CPU~0 starts executing the \co{b = a + 1}.
\item	CPU~0 receives the cache line from CPU~1, which still has
	a value of zero for \qco{a}.
\item	CPU~0 loads \qco{a} from its cache, finding the value zero.
	\label{item:app:whymb:Need Store Buffer}
\item	CPU~0 applies the entry from its store buffer to the newly
	arrived cache line, setting the value of \qco{a} in its cache
	to one.
\item	CPU~0 adds one to the value zero loaded for \qco{a} above,
	and stores it into the cache line containing \qco{b}
	(which we will assume is already owned by CPU~0).
\item	CPU~0 executes \co{assert(b == 2)}, which fails.
\end{sequence}

The problem is that we have two copies of \qco{a}, one in the cache and
the other in the store buffer.

This example breaks a very important guarantee, namely that each CPU
will always see its own operations as if they happened in program order.
Breaking this guarantee is violently counter-intuitive to software types,
so much so
that the hardware guys took pity and implemented ``store forwarding'',
where each CPU refers to (or ``snoops'') its store buffer as well
as its cache when performing loads, as shown in
\cref{fig:app:whymb:Caches With Store Forwarding}.
In other words, a given CPU's stores are directly forwarded to its
subsequent loads, without having to pass through the cache.

\begin{figure}
\centering
\resizebox{3in}{!}{\includegraphics{appendix/whymb/cacheSBf}}
\caption{Caches With Store Forwarding}
\label{fig:app:whymb:Caches With Store Forwarding}
\end{figure}

With store forwarding in place, item~\ref{item:app:whymb:Need Store Buffer}
in the above sequence would have found the correct value of 1 for \qco{a} in
the store buffer, so that the final value of \qco{b} would have been 2,
as one would hope.

\subsection{Store Buffers and Memory Barriers}
\label{sec:app:whymb:Store Buffers and Memory Barriers}

To see the second complication, a violation of global memory ordering,
consider the following code sequences
with variables \qco{a} and \qco{b} initially zero:

\begin{VerbatimN}[fontsize=\footnotesize,samepage=true]
void foo(void)
{
	a = 1;
	b = 1;
}

void bar(void)
{
	while (b == 0) continue;
	assert(a == 1);
}
\end{VerbatimN}

Suppose CPU~0 executes \co{foo()} and CPU~1 executes \co{bar()}.
Suppose further that the cache line containing \qco{a} resides only in CPU~1's
cache, and that the cache line containing \qco{b} is owned by CPU~0.
Then the sequence of operations might be as follows:
\begin{sequence}
\item	CPU~0 executes \co{a = 1}.
	The cache line is not in CPU~0's cache, so CPU~0 places the new
	value of \qco{a} in its store buffer and transmits a ``read
	invalidate'' message.
	\label{seq:app:whymb:Store Buffers and Memory Barriers}
\item	CPU~1 executes \co{while (b == 0) continue}, but the cache line
	containing \qco{b} is not in its cache.
	It therefore transmits a ``read'' message.
\item	CPU~0 executes \co{b = 1}.
	It already owns this cache line (in other words, the cache line
	is already in either the ``modified'' or the ``exclusive'' state),
	so it stores the new value of \qco{b} in its cache line.
\item	CPU~0 receives the ``read'' message, and transmits the
	cache line containing the now-updated value of \qco{b}
	to CPU~1, also marking the line as ``shared'' in its own cache
	(but only after writing back the line containing \qco{b} to main
	memory).
	\label{seq:app:whymb:Store Buffers and Memory Barriers store}
\item	CPU~1 receives the cache line containing \qco{b} and installs
	it in its cache.
\item	CPU~1 can now finish executing \co{while (b == 0) continue},
	and since it finds that the value of \qco{b} is 1, it proceeds
	to the next statement.
\item	CPU~1 executes the \co{assert(a == 1)}, and, since CPU~1 is
	working with the old value of \qco{a}, this assertion fails.
\item	CPU~1 receives the ``read invalidate'' message, and
	transmits the cache line containing \qco{a} to CPU~0 and
	invalidates this cache line from its own cache.
	But it is too late.
\item	CPU~0 receives the cache line containing \qco{a} and applies
	the buffered store just in time to fall victim to CPU~1's
	failed assertion.
	\label{seq:app:whymb:Store Buffers and Memory Barriers victim}
\end{sequence}

\EQuickQuiz{
	In \cref{seq:app:whymb:Store Buffers and Memory Barriers} above,
	why does CPU~0 need to issue a ``read invalidate''
	rather than a simple ``invalidate''?
	After all, \co{foo()} will overwrite the variable \co{a} in any
	case, so why should it care about the old value of \co{a}?
}\EQuickQuizAnswer{
	Because the cache line in question contains more data than just the
	variable \co{a}.
	Issuing ``invalidate'' instead of the needed ``read invalidate''
	would cause that other data to be lost, which would constitute
	a serious bug in the hardware.
}\EQuickQuizEnd

\EQuickQuiz{
	In \cref{seq:app:whymb:Store Buffers and Memory Barriers store}
	above, don't systems avoid that store to memory?
}\EQuickQuizAnswer{
	Yes, they do.
	But to do so, they add states beyond the MESI quadruple that
	this example is working within.
}\EQuickQuizEnd

\EQuickQuiz{
	In \cref{seq:app:whymb:Store Buffers and Memory Barriers victim}
	above, did \co{bar()} read a stale value from \co{a}, or did
	its reads of \co{b} and \co{a} get reordered?
}\EQuickQuizAnswer{
	It could be either, depending on the hardware implementation.
	And it really does not matter which.
	After all, the \co{bar()} function's \co{assert()} cannot tell
	the difference!
}\EQuickQuizEnd

The hardware designers cannot help directly here, since the CPUs have
no idea which variables are related, let alone how they might be related.
Therefore, the hardware designers provide memory-barrier instructions
to allow the software to tell the CPU about such relations.
The program fragment must be updated to contain the memory barrier:

\begin{VerbatimN}[fontsize=\footnotesize,samepage=true]
void foo(void)
{
	a = 1;
	smp_mb();
	b = 1;
}

void bar(void)
{
	while (b == 0) continue;
	assert(a == 1);
}
\end{VerbatimN}

The memory barrier \co{smp_mb()} will cause the CPU to flush its store
buffer before applying each subsequent store to its variable's cache line.
The CPU could either simply stall until the store buffer was empty
before proceeding, or it could use the store buffer to hold subsequent
stores until all of the prior entries in the store buffer had been
applied.

With this latter approach the sequence of operations might be as follows:
\begin{sequence}
\item	CPU~0 executes \co{a = 1}.
	The cache line is not in CPU~0's cache, so CPU~0 places the new
	value of \qco{a} in its store buffer and transmits a ``read
	invalidate'' message.
\item	CPU~1 executes \co{while (b == 0) continue}, but the cache line
	containing \qco{b} is not in its cache.
	It therefore transmits a ``read'' message.
\item	CPU~0 executes \co{smp_mb()}, and marks all current store-buffer
	entries (namely, the \co{a = 1}).
\item	CPU~0 executes \co{b = 1}.
	It already owns this cache line (in other words, the cache line
	is already in either the ``modified'' or the ``exclusive'' state),
	but there is a marked entry in the store buffer.
	Therefore, rather than store the new value of \qco{b} in the
	cache line, it instead places it in the store buffer (but
	in an \emph{unmarked} entry).
\item	CPU~0 receives the ``read'' message, and transmits the
	cache line containing the original value of \qco{b}
	to CPU~1.
	It also marks its own copy of this cache line as ``shared''.
\item	CPU~1 receives the cache line containing \qco{b} and installs
	it in its cache.
\item	CPU~1 can now load the value of \qco{b},
	but since it finds that the value of \qco{b} is still 0, it repeats
	the \co{while} statement.
	The new value of \qco{b} is safely hidden in CPU~0's store buffer.
\item	CPU~1 receives the ``read invalidate'' message, and
	transmits the cache line containing \qco{a} to CPU~0 and
	invalidates this cache line from its own cache.
\item	CPU~0 receives the cache line containing \qco{a} and applies
	the buffered store, placing this line into the ``modified''
	state.
\item	Since the store to \qco{a} was the only
	entry in the store buffer that was marked by the \co{smp_mb()},
	CPU~0 can also store the new value of \qco{b}---except for the
	fact that the cache line containing \qco{b} is now in ``shared''
	state.
\item	CPU~0 therefore sends an ``invalidate'' message to CPU~1.
\item	CPU~1 receives the ``invalidate'' message, invalidates the
	cache line containing \qco{b} from its cache, and sends an
	``acknowledgement'' message to CPU~0.
\item	CPU~1 executes \co{while (b == 0) continue}, but the cache line
	containing \qco{b} is not in its cache.
	It therefore transmits a ``read'' message to CPU~0.
\item	CPU~0 receives the ``acknowledgement'' message, and puts
	the cache line containing \qco{b} into the ``exclusive'' state.
	CPU~0 now stores the new value of \qco{b} into the cache line.
\item	CPU~0 receives the ``read'' message, and transmits the
	cache line containing the new value of \qco{b}
	to CPU~1.
	It also marks its own copy of this cache line as ``shared''.%
	\label{seq:app:whymb:Store buffers: All copies shared}
\item	CPU~1 receives the cache line containing \qco{b} and installs
	it in its cache.
\item	CPU~1 can now load the value of \qco{b},
	and since it finds that the value of \qco{b} is 1, it
	exits the \co{while} loop and proceeds
	to the next statement.
\item	CPU~1 executes the \co{assert(a == 1)}, but the cache line containing
	\qco{a} is no longer in its cache.
	Once it gets this cache from CPU~0, it will be
	working with the up-to-date value of \qco{a}, and the assertion
	therefore passes.
\end{sequence}

\QuickQuiz{
	After \cref{seq:app:whymb:Store buffers: All copies shared}
	in \cref{sec:app:whymb:Store Buffers and Memory Barriers} on
	\cpageref{seq:app:whymb:Store buffers: All copies shared},
	both CPUs might drop the cache line containing the new value of
	\qco{b}.
	Wouldn't that cause this new value to be lost?
}\QuickQuizAnswer{
	It might, and that is why real hardware takes steps to avoid
	this problem.
	A traditional approach, pointed out by Vasilevsky Alexander,
	is to write this cache line back to main memory before marking
	the cache line as ``shared''.
	A more efficient (though more complex) approach is to use
	additional state to indicate whether or not the cache line
	is ``dirty'', allowing the writeback to happen.
	Year-2000 systems went further, using much more state in order to
	avoid redundant writebacks~\cite[Figure 8.42]{DavidECuller1999}.
	It would be reasonable to assume that complexity has not decreased
	in the meantime.
}\QuickQuizEnd

As you can see, this process involves no small amount of bookkeeping.
Even something intuitively simple, like ``load the value of a'' can
involve lots of complex steps in silicon.

\section{Store Sequences Result in Unnecessary Stalls}
\label{sec:app:whymb:Store Sequences Result in Unnecessary Stalls}

Unfortunately, each store buffer must be relatively small, which means
that a CPU executing a modest sequence of stores can fill its store
buffer (for example, if all of them result in cache misses).
At that point, the CPU must once again wait for \IXpl{invalidation} to complete
in order to drain its store buffer before it can continue executing.
This same situation can arise immediately after a memory barrier, when
\emph{all} subsequent store instructions must wait for invalidations to
complete, regardless of whether or not these stores result in cache misses.

This situation can be improved by making invalidate acknowledge
messages arrive more quickly.
One way of accomplishing this is to use per-CPU queues of
invalidate messages, or ``invalidate queues''.

\subsection{Invalidate Queues}
\label{sec:app:whymb:Invalidate Queues}

One reason that invalidate acknowledge messages can take so long
is that they must ensure that the corresponding cache line is
actually invalidated, and this invalidation can be delayed if
the cache is busy, for example, if the CPU is intensively loading
and storing data, all of which resides in the cache.
In addition, if a large number of invalidate messages arrive
in a short time period, a given CPU might fall behind in processing
them, thus possibly stalling all the other CPUs.

However, the CPU need not actually invalidate the cache line
before sending the acknowledgement.
It could instead queue the invalidate message with the understanding
that the message will be processed before the CPU sends any further
messages regarding that cache line.

\subsection{Invalidate Queues and Invalidate Acknowledge}
\label{sec:app:whymb:Invalidate Queues and Invalidate Acknowledge}

\Cref{fig:app:whymb:Caches With Invalidate Queues}
shows a system with invalidate queues.
A CPU with an invalidate queue may acknowledge an invalidate message
as soon as it is placed in the queue, instead of having to wait until
the corresponding line is actually invalidated.
Of course, the CPU must refer to its invalidate queue when preparing
to transmit invalidation messages---if an entry for the corresponding
cache line is in the invalidate queue, the CPU cannot immediately
transmit the invalidate message; it must instead wait until the
invalidate-queue entry has been processed.

\begin{figure}
\centering
\resizebox{3in}{!}{\includegraphics{appendix/whymb/cacheSBfIQ}}
\caption{Caches With Invalidate Queues}
\label{fig:app:whymb:Caches With Invalidate Queues}
\end{figure}

Placing an entry into the invalidate queue is essentially a promise
by the CPU to process that entry before transmitting any MESI protocol
messages regarding that cache line.
As long as the corresponding data structures are not highly contended,
the CPU will rarely be inconvenienced by such a promise.

However, the fact that invalidate messages can be buffered in the
invalidate queue provides additional opportunity for memory-misordering,
as discussed in the next section.

\subsection{Invalidate Queues and Memory Barriers}
\label{sec:app:whymb:Invalidate Queues and Memory Barriers}

Let us suppose that CPUs queue invalidation requests, but respond to
them immediately.
This approach minimizes the \IXh{cache-invalidation}{latency} seen by CPUs
doing stores, but can defeat memory barriers, as seen in the following
example.

Suppose the values of \qco{a} and \qco{b} are initially zero,
that \qco{a} is replicated read-only (MESI ``shared'' state),
and that \qco{b}
is owned by CPU~0 (MESI ``exclusive'' or ``modified'' state).
Then suppose that CPU~0 executes \co{foo()} while CPU~1 executes
function \co{bar()} in the following code fragment:

\begin{fcvlabel}[ln:app:whymb:Breaking mb]
\begin{VerbatimN}[fontsize=\footnotesize,samepage=true,commandchars=\\\[\]]
void foo(void)
{
	a = 1;
	smp_mb();	\lnlbl[mb]
	b = 1;
}

void bar(void)
{
	while (b == 0) continue;
	assert(a == 1);
}
\end{VerbatimN}
\end{fcvlabel}

Then the sequence of operations might be as follows:
\begin{fcvref}[ln:app:whymb:Breaking mb]
\begin{sequence}
\item	CPU~0 executes \co{a = 1}.
	The corresponding cache line is read-only in CPU~0's cache, so
	CPU~0 places the new value of \qco{a} in its store buffer and
	transmits an ``invalidate'' message in order to flush the
	corresponding cache line from CPU~1's cache.
	\label{seq:app:whymb:Invalidate Queues and Memory Barriers}
\item	CPU~1 executes \co{while (b == 0) continue}, but the cache line
	containing \qco{b} is not in its cache.
	It therefore transmits a ``read'' message.
\item	CPU~1 receives CPU~0's ``invalidate'' message, queues it, and
	immediately responds to it.
\item	CPU~0 receives the response from CPU~1, and is therefore free
	to proceed past the \co{smp_mb()} on \clnref{mb} above, moving
	the value of \qco{a} from its store buffer to its cache line.
\item	CPU~0 executes \co{b = 1}.
	It already owns this cache line (in other words, the cache line
	is already in either the ``modified'' or the ``exclusive'' state),
	so it stores the new value of \qco{b} in its cache line.
\item	CPU~0 receives the ``read'' message, and transmits the
	cache line containing the now-updated value of \qco{b}
	to CPU~1, also marking the line as ``shared'' in its own cache.
\item	CPU~1 receives the cache line containing \qco{b} and installs
	it in its cache.
\item	CPU~1 can now finish executing \co{while (b == 0) continue},
	and since it finds that the value of \qco{b} is 1, it proceeds
	to the next statement.
\item	CPU~1 executes the \co{assert(a == 1)}, and, since the
	old value of \qco{a} is still in CPU~1's cache,
	this assertion fails.
\item	Despite the assertion failure, CPU~1 processes the queued
	``invalidate'' message, and (tardily)
	invalidates the cache line containing \qco{a} from its own cache.
\end{sequence}
\end{fcvref}

\QuickQuiz{
	In \cref{seq:app:whymb:Invalidate Queues and Memory Barriers}
	of the first scenario in
	\cref{sec:app:whymb:Invalidate Queues and Memory Barriers},
	why is an ``invalidate'' sent instead of a ''read invalidate''
	message?
	Doesn't CPU~0 need the values of the other variables that share
	this cache line with \qco{a}?
}\QuickQuizAnswer{
	CPU~0 already has the values of these variables, given that it
	has a read-only copy of the cache line containing \qco{a}.
	Therefore, all CPU~0 need do is to cause the other CPUs to discard
	their copies of this cache line.
	An ``invalidate'' message therefore suffices.
}\QuickQuizEnd

There is clearly not much point in accelerating invalidation responses
if doing so causes memory barriers to effectively be ignored.
However, the memory-barrier instructions can interact with
the invalidate queue, so that when a given CPU executes a memory
barrier, it marks all the entries currently in its invalidate queue,
and forces any subsequent load to wait until all marked entries
have been applied to the CPU's cache.
Therefore, we can add a memory barrier to function \co{bar} as follows:

\begin{fcvlabel}[ln:app:whymb:Add mb]
\begin{VerbatimN}[fontsize=\footnotesize,samepage=true,commandchars=\\\[\]]
void foo(void)
{
	a = 1;
	smp_mb();		\lnlbl[mb1]
	b = 1;
}

void bar(void)
{
	while (b == 0) continue;
	smp_mb();
	assert(a == 1);
}
\end{VerbatimN}
\end{fcvlabel}

\QuickQuiz{
	Say what???
	Why do we need a memory barrier here, given that the CPU cannot
	possibly execute the \co{assert()} until after the
	\co{while} loop completes?
}\QuickQuizAnswer{
	Suppose that memory barrier was omitted.

	Keep in mind that CPUs are free to speculatively execute later
	loads, which can have the effect of executing the assertion
	before the \co{while} loop completes.
	Furthermore, compilers assume that only the currently executing
	thread is updating the variables, and this assumption allows
	the compiler to hoist the load of \co{a} to precede the
	loop.

	In fact, some compilers would transform the loop to a branch
	around an infinite loop as follows:

\begin{VerbatimN}[fontsize=\footnotesize,samepage=true]
void foo(void)
{
	a = 1;
	smp_mb();
	b = 1;
}

void bar(void)
{
	if (b == 0)
		for (;;)
			continue;
	assert(a == 1);
}
\end{VerbatimN}

	Given this optimization, the code would behave in a completely
	different way than the original code.
	If \co{bar()} observed \qco{b == 0}, the assertion could of
	course not be reached at all due to the infinite loop.
	However, if \co{bar()} loaded the value \qco{1} just as
	\qco{foo()} stored it, the CPU might still have the old
	zero value of \qco{a} in its cache, which would cause
	the assertion to fire.
	You should of course use volatile casts (for example, those
	volatile casts implied by the C11 relaxed atomic load operation)
	to prevent the compiler from optimizing your parallel code
	into oblivion.
	But volatile casts would not prevent a weakly ordered CPU
	from loading the old value for \qco{a} from its cache, which
	means that this code also requires the explicit memory barrier
	in \qco{bar()}.

	In short, both compilers and CPUs aggressively apply
	code-reordering optimizations, so you must clearly communicate
	your constraints using the compiler directives and memory barriers
	provided for this purpose.
%
}\QuickQuizEnd

\begin{fcvref}[ln:app:whymb:Add mb]
With this change, the sequence of operations might be as follows:
\begin{sequence}
\item	CPU~0 executes \co{a = 1}.
	The corresponding cache line is read-only in CPU~0's cache,
	so CPU~0 places the new value of \qco{a} in its store buffer and
	transmits an ``invalidate'' message in order to flush the
	corresponding cache line from CPU~1's cache.
\item	CPU~1 executes \co{while (b == 0) continue}, but the cache line
	containing \qco{b} is not in its cache.
	It therefore transmits a ``read'' message.
\item	CPU~1 receives CPU~0's ``invalidate'' message, queues it, and
	immediately responds to it.
\item	CPU~0 receives the response from CPU~1, and is therefore free
	to proceed past the \co{smp_mb()} on \clnref{mb1} above, moving
	the value of \qco{a} from its store buffer to its cache line.
\item	CPU~0 executes \co{b = 1}.
	It already owns this cache line (in other words, the cache line
	is already in either the ``modified'' or the ``exclusive'' state),
	so it stores the new value of \qco{b} in its cache line.
\item	CPU~0 receives the ``read'' message, and transmits the
	cache line containing the now-updated value of \qco{b}
	to CPU~1, also marking the line as ``shared'' in its own cache.
\item	CPU~1 receives the cache line containing \qco{b} and installs
	it in its cache.
\item	CPU~1 can now finish executing \co{while (b == 0) continue},
	and since it finds that the value of \qco{b} is 1, it proceeds
	to the next statement, which is now a memory barrier.
\item	CPU~1 must now stall until it processes all pre-existing
	messages in its invalidation queue.
\item	CPU~1 now processes the queued
	``invalidate'' message, and
	invalidates the cache line containing \qco{a} from its own cache.
\item	CPU~1 executes the \co{assert(a == 1)}, and, since the
	cache line containing \qco{a} is no longer in CPU~1's cache,
	it transmits a ``read'' message.
\item	CPU~0 responds to this ``read'' message with the cache line
	containing the new value of \qco{a}.
\item	CPU~1 receives this cache line, which contains a value of 1 for
	\qco{a}, so that the assertion does not trigger.
\end{sequence}
\end{fcvref}

With much passing of MESI messages, the CPUs arrive at the correct answer.
This section illustrates why CPU designers must be extremely careful
with their cache-coherence optimizations.
The key requirement is that the memory barriers provide the appearance
of ordering to the software.
As long as these appearances are maintained, the hardware can carry
out whatever queueing, buffering, marking, stallings, and flushing
optimizations it likes.

\QuickQuiz{
	Instead of all of this marking of invalidation-queue entries
	and stalling of loads, why not simply force an immediate flush
	of the invalidation queue?
}\QuickQuizAnswer{
	An immediate flush of the invalidation queue would do the trick.
	Except that the common-case super-scalar CPU is executing many
	instructions at once, and not necessarily even in the expected
	order.
	So what would ``immediate'' even mean?
	The answer is clearly ``not much''.

	Nevertheless, for simpler CPUs that execute instructions serially,
	flushing the invalidation queue might be a reasonable implementation
	strategy.
}\QuickQuizEnd

\section{Read and Write Memory Barriers}
\label{sec:app:whymb:Read and Write Memory Barriers}

In the previous section, memory barriers were used to mark entries in both
the store buffer and the invalidate queue.
But in our code fragment, \co{foo()} had no reason to do anything with the
invalidate queue, and \co{bar()} similarly had no reason to do anything
with the store buffer.

Many CPU architectures therefore provide weaker memory-barrier
instructions that do only one or the other of these two.
Roughly speaking, a ``\IXBh{read}{memory barrier}'' marks only the invalidate
queue (and snoops entries in the store buffer) and a ``\IXBh{write}{memory
barrier}'' marks only the store buffer, while a full-fledged memory
barrier does all of the above.

The software-visible effect of these hardware mechanisms is that a read
memory barrier orders only loads on the CPU that executes it, so that
all loads preceding the read memory barrier will appear to have completed
before any load following the read memory barrier.
Similarly, a write memory barrier orders only stores, again on the
CPU that executes it, and again so that all stores preceding the write
memory barrier will appear to have completed before any store following
the write memory barrier.
A full-fledged memory barrier orders both loads and stores, but again
only on the CPU executing the memory barrier.

\QuickQuiz{
	But can't full memory barriers impose global ordering?
	After all, isn't that needed to provide the ordering
	shown in \cref{lst:formal:IRIW Litmus Test}?
}\QuickQuizAnswer{
	Sort of.

	Note well that this litmus test has not one but two full
	memory-barrier instructions, namely the two \co{sync} instructions
	executed by \co{P2} and \co{P3}.

	It is the interaction of those two instructions that provides
	the global ordering, not just their individual execution.
	For example, each of those two \co{sync} instructions might stall
	waiting for all CPUs to process their invalidation queues before
	allowing subsequent instructions to execute.\footnote{
		Real-life hardware of course applies many optimizations
		to minimize the resulting stalls.}
}\QuickQuizEnd

If we update \co{foo()} and \co{bar()} to use read and write memory
barriers, they appear as follows:

\begin{VerbatimN}[fontsize=\footnotesize,samepage=true]
void foo(void)
{
	a = 1;
	smp_wmb();
	b = 1;
}

void bar(void)
{
	while (b == 0) continue;
	smp_rmb();
	assert(a == 1);
}
\end{VerbatimN}

Some computers have even more flavors of memory barriers, but
understanding these three variants will provide a good introduction
to memory barriers in general.

\section{Example Memory-Barrier Sequences}
\label{sec:app:whymb:Example Memory-Barrier Sequences}

This section presents some seductive but subtly broken uses of
memory barriers.
Although many of them will work most of the time, and some will
work all the time on some specific CPUs, these uses must be avoided
if the goal is to produce code that works reliably on all CPUs.
To help us better see the subtle breakage, we first need to focus
on an ordering-hostile architecture.

\subsection{Ordering-Hostile Architecture}
\label{sec:app:whymb:Ordering-Hostile Architecture}

A number of ordering-hostile computer systems have been produced over
the decades,
but the nature of the hostility has always been extremely subtle,
and understanding it has required detailed knowledge of the specific
hardware.
Rather than picking on a specific hardware vendor, and as a presumably
attractive alternative to dragging the reader through detailed
technical specifications, let us instead design a mythical but maximally
memory-ordering-hostile computer architecture.\footnote{
	Readers preferring a detailed look at real hardware
	architectures are encouraged to consult CPU vendors'
	manuals~\cite{ALPHA95,AMDOpteron02,IntelItanium02v2,PowerPC94,MichaelLyons05a,SPARC94,IntelXeonV3-96a,IntelXeonV2b-96a,IBMzSeries04a},
	Gharachorloo's dissertation~\cite{Gharachorloo95},
	Peter Sewell's work~\cite{PeterSewell2021weakmemory}, or
	the excellent hardware-oriented primer by
	Sorin, Hill, and Wood~\cite{DanielJSorin2011MemModel}.}

This hardware must obey the following ordering
constraints~\cite{PaulMcKenney2005i,PaulMcKenney2005j}:
\begin{enumerate}
\item	Each CPU will always perceive its own memory accesses
	as occurring in program order.
\item	CPUs will reorder a given operation with a store only
	if the two operations are referencing different locations.
\item	All of a given CPU's loads preceding a read memory barrier
	(\co{smp_rmb()}) will be perceived by all CPUs to precede
	any loads following that read memory barrier.
\item	All of a given CPU's stores preceding a write memory barrier
	(\co{smp_wmb()}) will be perceived by all CPUs to precede
	any stores following that write memory barrier.
\item	All of a given CPU's accesses (loads and stores) preceding a
	full memory barrier
	(\co{smp_mb()}) will be perceived by all CPUs to precede
	any accesses following that memory barrier.
\end{enumerate}

\QuickQuiz{
	Does the guarantee that each CPU sees its own memory accesses
	in order also guarantee that each user-level thread will see
	its own memory accesses in order?
	Why or why not?
}\QuickQuizAnswer{
	No.
	Consider the case where a thread migrates from one CPU to
	another, and where the destination CPU perceives the source
	CPU's recent memory operations out of order.
	To preserve user-mode sanity, kernel hackers must use memory
	barriers in the context-switch path.
	However, the locking already required to safely do a context
	switch should automatically provide the memory barriers needed
	to cause the user-level task to see its own accesses in order.
	That said, if you are designing a super-optimized scheduler,
	either in the kernel or at user level,
	please keep this scenario in mind!
}\QuickQuizEnd

Imagine a large \IXacrf{nuca} system that,
in order to provide fair allocation
of interconnect bandwidth to CPUs in a given node, provided per-CPU
queues in each node's interconnect interface, as shown in
\cref{fig:app:whymb:Example Ordering-Hostile Architecture}.
Although a given CPU's accesses are ordered as specified by memory
barriers executed by that CPU, however, the relative order of a
given pair of CPUs' accesses could be severely reordered,
as we will see.\footnote{
	Any real hardware architect or designer will no doubt be
	objecting strenuously,
	as they just might be a bit upset about the prospect of working
	out which queue should handle a message involving a cache line
	that both CPUs accessed, to say nothing of the many races that
	this example poses.
	All I can say is ``Give me a better example''.}

\begin{figure}
\centering
\resizebox{3in}{!}{\includegraphics{appendix/whymb/hostileordering}}
\caption{Example Ordering-Hostile Architecture}
\label{fig:app:whymb:Example Ordering-Hostile Architecture}
\end{figure}

\subsection{Example 1}
\label{sec:app:whymb:Example 1}

\Cref{lst:app:whymb:Memory Barrier Example 1}
shows three code fragments, executed concurrently by CPUs~0, 1, and~2.
Each of \qco{a}, \qco{b}, and \qco{c} are initially zero.

\floatstyle{plaintop}
\restylefloat{listing}

\begin{listing}
\scriptsize
\centering{\tt
\begin{tabular}{l|l|l}
	\multicolumn{1}{c|}{\nf{CPU~0}} &
		\multicolumn{1}{c|}{\nf{CPU~1}} &
			\multicolumn{1}{c}{\nf{CPU~2}} \\
	\hline
	\hline
	a = 1;		 &		& \\
	\tco{smp_wmb();} & while (b == 0); & \\
	b = 1;		 & c = 1;	& z = c; \\
			 &		& \tco{smp_rmb();} \\
			 &		& x = a; \\
			 &		& assert(z == 0 || x == 1); \\
\end{tabular}}
\caption{Memory Barrier Example 1}
\label{lst:app:whymb:Memory Barrier Example 1}
\end{listing}

Suppose CPU~0 recently experienced many cache misses, so that its
message queue is full, but that CPU~1 has been running exclusively within
the cache, so that its message queue is empty.
Then CPU~0's assignment to \qco{a} and \qco{b} will appear in Node~0's cache
immediately (and thus be visible to CPU~1), but will be blocked behind
CPU~0's prior traffic.
In contrast, CPU~1's assignment to \qco{c} will sail through CPU~1's
previously empty queue.
Therefore, CPU~2 might well see CPU~1's assignment to \qco{c} before
it sees CPU~0's assignment to \qco{a}, causing the assertion to fire,
despite the memory barriers.

Therefore, portable code cannot rely on this assertion not firing,
as both the compiler and the CPU can reorder the code so as to trip
the assertion.

\QuickQuiz{
	Could this code be fixed by inserting a memory barrier
	between CPU~1's \qco{while} and assignment to \qco{c}?
	Why or why not?
}\QuickQuizAnswer{
	No.
	Such a memory barrier would only force ordering local to CPU~1.
	It would have no effect on the relative ordering of CPU~0's and
	CPU~1's accesses, so the assertion could still fail.
	However, all mainstream computer systems provide one mechanism
	or another to provide ``transitivity'', which provides
	intuitive causal ordering:
	If B saw the effects of A's accesses, and C saw the effects of
	B's accesses, then C must also see the effects of A's accesses.
	In short, hardware designers have taken at least a little pity
	on software developers.
}\QuickQuizEnd

\subsection{Example 2}
\label{sec:app:whymb:Example 2}

\Cref{lst:app:whymb:Memory Barrier Example 2}
shows three code fragments, executed concurrently by CPUs~0, 1, and~2.
Both \qco{a} and \qco{b} are initially zero.

\begin{listing}
\scriptsize
\centering{\tt
\begin{tabular}{l|l|l}
	\multicolumn{1}{c|}{\nf{CPU~0}} &
		\multicolumn{1}{c|}{\nf{CPU~1}} &
			\multicolumn{1}{c}{\nf{CPU~2}} \\
	\hline
	\hline
	a = 1;	     & while (a == 0); & \\
		     & \tco{smp_mb();}	& y = b; \\
		     & b = 1;		& \tco{smp_rmb();} \\
		     &			& x = a; \\
		     &			& assert(y == 0 || x == 1); \\
\end{tabular}}
\caption{Memory Barrier Example 2}
\label{lst:app:whymb:Memory Barrier Example 2}
\end{listing}

Again, suppose CPU~0 recently experienced many cache misses, so that its
message queue is full, but that CPU~1 has been running exclusively within
the cache, so that its message queue is empty.
Then CPU~0's assignment to \qco{a} will appear in Node~0's cache
immediately (and thus be visible to CPU~1), but will be blocked behind
CPU~0's prior traffic.
In contrast, CPU~1's assignment to \qco{b} will sail through CPU~1's
previously empty queue.
Therefore, CPU~2 might well see CPU~1's assignment to \qco{b} before
it sees CPU~0's assignment to \qco{a}, causing the assertion to fire,
despite the memory barriers.

In theory, portable code should not rely on this example code fragment,
however, as before, in practice it actually does work on most
mainstream computer systems.

\subsection{Example 3}
\label{sec:app:whymb:Example 3}

\Cref{lst:app:whymb:Memory Barrier Example 3}
shows three code fragments, executed concurrently by CPUs~0, 1, and~2.
All variables are initially zero.

\begin{listing*}
\scriptsize
\centering{\tt
\begin{tabular}{r|l|l|l}
	& \multicolumn{1}{c|}{\nf{CPU~0}} &
		\multicolumn{1}{c|}{\nf{CPU~1}} &
			\multicolumn{1}{c}{\nf{CPU~2}} \\
	\hline
	\hline
 1 &	a = 1; &			& \\
 2 &	\tco{smp_wmb();}&		& \\
 3 &	b = 1;		& while (b == 0); & while (b == 0); \\
 4 &			& \tco{smp_mb();}& \tco{smp_mb();} \\
 5 &			& c = 1;	& d = 1; \\
 6 &	while (c == 0); &		& \\
 7 &	while (d == 0); &		& \\
 8 &	\tco{smp_mb();}	&		& \\
 9 &	e = 1; &			& assert(e == 0 || a == 1); \\
\end{tabular}}
\caption{Memory Barrier Example 3}
\label{lst:app:whymb:Memory Barrier Example 3}
\end{listing*}

\floatstyle{ruled}
\restylefloat{listing}

Note that neither CPU~1 nor CPU~2 can proceed to line~5 until they see
CPU~0's assignment to \qco{b} on line~3.
Once CPU~1 and~2 have executed their memory barriers on line~4, they
are both guaranteed to see all assignments by CPU~0 preceding its memory
barrier on line~2.
Similarly, CPU~0's memory barrier on line~8 pairs with those of CPUs~1 and~2
on line~4, so that CPU~0 will not execute the assignment to \qco{e} on
line~9 until after its assignment to \qco{b} is visible to both of the
other CPUs.
Therefore, CPU~2's assertion on line~9 is guaranteed \emph{not} to fire.

\QuickQuizSeries{%
\QuickQuizB{
	Suppose that lines~3--5 for CPUs~1 and~2 in
	\cref{lst:app:whymb:Memory Barrier Example 3}
	are in an interrupt
	handler, and that the CPU~2's line~9 runs at process level.
	In other words, the code in all three columns of the table
	runs on the same CPU, but the first two columns run in an
	interrupt handler, and the third column runs at process
	level, so that the code in third column can be interrupted
	by the code in the first two columns.
	What changes, if any, are required to enable the code to work
	correctly, in other words, to prevent the assertion from firing?
}\QuickQuizAnswerB{
	The assertion must ensure that the load of
	\qco{e} precedes that of \qco{a}.
	In the Linux kernel, the \co{barrier()} primitive may be used to
	accomplish this in much the same way that the memory barrier was
	used in the assertions in the previous examples.
	For example, the assertion can be modified as follows:

\begin{VerbatimU}[fontsize=\footnotesize]
r1 = e;
barrier();
assert(r1 == 0 || a == 1);
\end{VerbatimU}

	No changes are needed to the code in the first two columns,
	because interrupt handlers run atomically from the perspective
	of the interrupted code.
}\QuickQuizEndB
%
\QuickQuizE{
	If CPU~2 executed an \co{assert(e==0||c==1)} in the example in
	\cref{lst:app:whymb:Memory Barrier Example 3},
	would this assert ever trigger?
}\QuickQuizAnswerE{
	The result depends on whether the CPU supports ``transitivity''.
	In other words, CPU~0 stored to \qco{e} after seeing CPU~1's
	store to \qco{c}, with a memory barrier between CPU~0's load
	from \qco{c} and store to \qco{e}.
	If some other CPU sees CPU~0's store to \qco{e}, is it also
	guaranteed to see CPU~1's store?

	All CPUs I am aware of claim to provide transitivity.
}\QuickQuizEndE
}

The Linux kernel's \co{synchronize_rcu()} primitive uses an algorithm
similar to that shown in this example.

\section{Are Memory Barriers Forever?}
\label{sec:app:whymb:Are Memory Barriers Forever?}

There have been a number of recent systems that are significantly less
aggressive about out-of-order execution in general and re-ordering
memory references in particular.
Will this trend continue to the point where memory barriers are a thing
of the past?

The argument in favor would cite proposed massively multi-threaded hardware
architectures, so that each thread would wait until memory was ready,
with tens, hundreds, or even thousands of other threads making progress
in the meantime.
In such an architecture, there would be no need for memory barriers,
because a given thread would simply wait for all outstanding operations
to complete before proceeding to the next instruction.
Because there would be potentially thousands of other threads, the
CPU would be completely utilized, so no CPU time would be wasted.

The argument against would cite the extremely limited number of applications
capable of scaling up to a thousand threads, as well as increasingly
severe realtime requirements, which are in the tens of microseconds
for some applications.
The realtime-response requirements are difficult enough to meet as is,
and would be even more difficult to meet given the extremely low
single-threaded throughput implied by the massive multi-threaded
scenarios.

Another argument in favor would cite increasingly sophisticated
latency-hiding hardware implementation techniques that might well allow
the CPU to provide the illusion of fully sequentially consistent
execution while still providing almost all of the performance advantages
of out-of-order execution.
A counter-argument would cite the increasingly severe power-efficiency
requirements presented both by battery-operated devices and by
environmental responsibility.

Who is right?
We have no clue, so we are preparing to live with either scenario.

\section{Advice to Hardware Designers}
\label{sec:app:whymb:Advice to Hardware Designers}

There are any number of things that hardware designers can do
to make the lives of software people difficult.
Here is a list of a few such things that we have encountered in
the past, presented here in the hope that it might help prevent
future such problems:
\begin{enumerate}
\item	I/O devices that ignore cache coherence.

	This charming misfeature can result in DMAs from memory
	missing recent changes to the output buffer, or, just as
	bad, cause input buffers to be overwritten by the contents
	of CPU caches just after the DMA completes.
	To make your system work in face of such misbehavior,
	you must carefully flush the CPU caches of any location
	in any DMA buffer before presenting that buffer to the
	I/O device.
	Otherwise, a store from one of the CPUs might not be
	accounted for in the data DMAed out through the device.
	This is a form of data corruption, which is an extremely
	serious bug.

	Similarly, you need to invalidate\footnote{
		Why not flush?
		If there is a difference, then a CPU must have incorrectly
		stored to the DMA buffer in the midst of the DMA operation.}
	the CPU caches corresponding to any location in any DMA buffer
	after DMA to that buffer completes.
	Otherwise, a given CPU might see the old data still residing in
	its cache instead of the newly DMAed data that it was supposed
	to see.
	This is another form of data corruption.

	And even then, you need to be \emph{very} careful to avoid
	pointer bugs, as even a misplaced read to an input buffer
	can result in corrupting the data input!
	One way to avoid this is to invalidate all of the caches of
	all of the CPUs once the DMA completes, but it is much easier
	and more efficient if the device DMA participates in the
	cache-coherence protocol, making all of this flushing and
	invalidating unnecessary.

\item	External busses that fail to transmit cache-coherence data.

	This is an even more painful variant of the above problem,
	but causes groups of devices---and even memory itself---to
	fail to respect cache coherence.
	It is my painful duty to inform you that as embedded systems
	move to multicore architectures, we will no doubt see a fair
	number of such problems arise.
	By the year 2021, there were some efforts to address
	these problems with new interconnect standards, with some
	debate as to how effective these standards will really
	be~\cite{WilliamGWong2019CCIX-CXL}.

\item	Device interrupts that ignore cache coherence.

	This might sound innocent enough---after all, interrupts
	aren't memory references, are they?
	But imagine a CPU with a split cache, one bank of which is
	extremely busy, therefore holding onto the last cacheline
	of the input buffer.
	If the corresponding I/O-complete interrupt reaches this
	CPU, then that CPU's memory reference to the last cache
	line of the buffer could return old data, again resulting
	in data corruption, but in a form that will be invisible
	in a later crash dump.
	By the time the system gets around to dumping the offending
	input buffer, the DMA will most likely have completed.

\item	\IXAcrfpl{ipi} that ignore cache coherence.

	This can be problematic if the IPI reaches its destination
	before all of the cache lines in the corresponding message
	buffer have been committed to memory.

\item	Context switches that get ahead of cache coherence.

	If memory accesses can complete too wildly out of order,
	then context switches can be quite harrowing.
	If the task flits from one CPU to another before all the
	memory accesses visible to the source CPU make it to the
	destination CPU, then the task could easily see the corresponding
	variables revert to prior values, which can fatally confuse
	most algorithms.

\item	Overly kind simulators and emulators.

	It is difficult to write simulators or emulators that force
	memory re-ordering, so software that runs just fine in
	these environments can get a nasty surprise when it first
	runs on the real hardware.
	Unfortunately, it is still the rule that the hardware is more
	devious than are the simulators and emulators, but we hope that
	this situation changes.
\end{enumerate}

Again, we encourage hardware designers to avoid these practices!

\QuickQuizAnswersChp{qqzwhymb}
