\begin{figure*}\centering
\ebresizewidth{
\begin{subcaptionblock}[t][][t]{2.1in}\centering
\resizebox{2.1in}{!}{\includegraphics{cartoons/1kHz}}
\caption{At 1\,kHz}
\label{fig:app:styleguide:At 1kHz}
\end{subcaptionblock}
\qquad
\begin{subcaptionblock}[t][][t]{2.3in}\centering
\resizebox{2.3in}{!}{\includegraphics{cartoons/100kHz}}
\caption{At 100\,kHz}
\label{fig:app:styleguide:At 100kHz}
\end{subcaptionblock}
}
\caption{Timer Wheels (subcaption)}
\label{fig:app:styleguide:Timer Wheels (subcaption)}
\end{figure*}

To prevent a pair of closely related figures or listings
from being placed in different pages, it is desirable to group
them into a single floating object.
The ``subcaption'' package provides the features to do so.\footnote{
  One problem of grouping figures might be the complexity in
  \LaTeX\ source.}

Two floating objects can be placed side by side by using
\co{\\parbox}, \co{minipage}, or \co{subcaptionblock}.
For example,
\cref{fig:advsync:Timer Wheel at 1kHz,fig:advsync:Timer Wheel at 100kHz}
can be grouped together by using a pair of \co{subcaptionblock}s
as shown in
\cref{fig:app:styleguide:Timer Wheels (subcaption)}.
