% appendix/questions/concurrentparallel.tex
% mainfile: ../../perfbook.tex
% SPDX-License-Identifier: CC-BY-SA-3.0

\section{What is the Difference Between ``Concurrent'' and ``Parallel''?}
\label{sec:app:questions:What is the Difference Between ``Concurrent'' and ``Parallel''?}

From a classic computing perspective, ``\IX{concurrent}'' and ``\IX{parallel}''
are clearly synonyms.
However, this has not stopped many people from drawing distinctions
between the two, and it turns out that these distinctions can be
understood from a couple of different perspectives.

The first perspective treats ``parallel'' as an abbreviation for
``data parallel'', and treats ``concurrent'' as pretty much everything
else.
From this perspective, in parallel computing, each partition of the
overall problem can proceed completely independently, with no
communication with other partitions.
In this case, little or no coordination among partitions is required.
In contrast, concurrent computing might well have tight interdependencies,
in the form of contended locks, transactions, or other synchronization
mechanisms.

\QuickQuiz{
	Suppose a portion of a program uses RCU read-side primitives
	as its only synchronization mechanism.
	Is this parallelism or concurrency?
}\QuickQuizAnswer{
	Yes.
}\QuickQuizEnd

This of course begs the question of why such a distinction matters,
which brings us to the second perspective, that of the underlying scheduler.
Schedulers come in a wide range of complexities and capabilities, and
as a rough rule of thumb, the more tightly and irregularly a set of
parallel processes communicate, the higher the level of sophistication
required from the scheduler.
As such, parallel computing's avoidance of interdependencies means that
parallel-computing programs run well on the least-capable schedulers.
In fact, a pure parallel-computing program can run successfully after
being arbitrarily subdivided and interleaved onto a uniprocessor.\footnote{
	Yes, this does mean that data-parallel-computing programs are
	best-suited for sequential execution.
	Why did you ask?}
In contrast, concurrent-computing programs might well require extreme
subtlety on the part of the scheduler.

One could argue that we should simply demand a reasonable level of
competence from the scheduler, so that we could simply ignore any
distinctions between parallelism and concurrency.
Although this is often a good strategy,
there are important situations where \IX{efficiency},
\IX{performance}, and \IX{scalability} concerns sharply limit the level
of competence that the scheduler can reasonably offer.
One important example is when the scheduler is implemented in
hardware, as it often is in SIMD units or \IXacrpl{gpgpu}.
Another example is a workload where the units of work are quite
short, so that even a software-based scheduler must make hard choices
between subtlety on the one hand and efficiency on the other.

Now, this second perspective can be thought of as making the workload
match the available scheduler, with parallel workloads able to
use simple schedulers and concurrent workloads requiring
sophisticated schedulers.

Unfortunately, this perspective does not always align with the
dependency-based distinction put forth by the first perspective.
For example, a highly interdependent lock-based workload
with one thread per CPU can make do with a trivial scheduler
because no scheduler decisions are required.
In fact, some workloads of this type can even be run one after another
on a sequential machine.
Therefore, such a workload would be labeled ``concurrent'' by the first
perspective and ``parallel'' by many taking the second perspective.

\QuickQuiz{
	In what part of the second (scheduler-based) perspective would
	the lock-based single-thread-per-CPU workload be considered
	``concurrent''?
}\QuickQuizAnswer{
	The people who would like to arbitrarily subdivide and interleave
	the workload.
	Of course, an arbitrary subdivision might end up separating
	a lock acquisition from the corresponding lock release, which
	would prevent any other thread from acquiring that lock.
	If the locks were pure spinlocks, this could even result in
	deadlock.
}\QuickQuizEnd

Which is just fine.
No rule that humankind writes carries any weight against the objective
universe, not even rules dividing multiprocessor programs into categories
such as ``concurrent'' and ``parallel''.

This categorization failure does not mean such rules are useless,
but rather that you should take on a suitably skeptical frame of mind when
attempting to apply them to new situations.
As always, use such rules where they apply and ignore them otherwise.

In fact, it is likely that new categories will arise in addition
to parallel, concurrent, map-reduce, task-based, and so on.
Some will stand the test of time, but good luck guessing which!
