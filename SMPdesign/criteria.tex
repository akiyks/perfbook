% SMPdesign/criteria.tex
% mainfile: ../perfbook.tex
% SPDX-License-Identifier: CC-BY-SA-3.0

\section{Design Criteria}
\label{sec:SMPdesign:Design Criteria}
%
\epigraph{One pound of learning requires ten pounds of commonsense to apply it.}
	 {\emph{Persian proverb}}

One way to obtain the best performance and scalability is to simply
hack away until you converge on the best possible parallel program.
Unfortunately, if your program is other than microscopically tiny,
the space of possible parallel programs is so huge
that convergence is not guaranteed in the lifetime of the universe.
Besides, what exactly is the ``best possible parallel program''?
After all, Section~\ref{sec:intro:Parallel Programming Goals}
called out no fewer than three parallel-programming goals of
\IX{performance}, \IX{productivity}, and \IX{generality},
and the best possible performance will likely come at a cost in
terms of productivity and generality.
We clearly need to be able to make higher-level choices at design
time in order to arrive at an acceptably good parallel program
before that program becomes obsolete.

However, more detailed design criteria are required to
actually produce a real-world design, a task taken up in this section.
This being the real world, these criteria often conflict to a
greater or lesser degree, requiring that the designer carefully
balance the resulting tradeoffs.

As such, these criteria may be thought of as the ``forces''
acting on the design, with particularly good tradeoffs between
these forces being called ``design patterns''~\cite{Alexander79,GOF95}.

The design criteria for attaining the three parallel-programming goals
are speedup,
contention, overhead, read-to-write ratio, and complexity:
\begin{description}
\item[Speedup:]  As noted in
	Section~\ref{sec:intro:Parallel Programming Goals},
	increased performance is the major reason
	to go to all of the time and trouble
	required to parallelize it.
	Speedup is defined to be the ratio of the time required
	to run a sequential version of the program to the time
	required to run a parallel version.
\item[Contention:]  If more CPUs are applied to a parallel
	program than can be kept busy by that program,
	the excess CPUs are prevented from doing
	useful work by contention.
	This may be \IX{lock contention}, memory contention, or a host
	of other performance killers.
\item[Work-to-Synchronization Ratio:]  A uniprocessor,
	single\-/threaded, non-preemptible, and non\-/interruptible\footnote{
		Either by masking interrupts or by being oblivious to them.}
	version of a given parallel
	program would not need any synchronization primitives.
	Therefore, any time consumed by these primitives
	(including communication cache misses as well as
	\IXh{message}{latency}, locking primitives, atomic instructions,
	and memory barriers)
	is overhead that does not contribute directly to the useful
	work that the program is intended to accomplish.
	Note that the important measure is the
	relationship between the synchronization overhead
	and the overhead of the code in the \IX{critical section}, with larger
	critical sections able to tolerate greater synchronization overhead.
	The work-to-synchronization ratio is related to
	the notion of synchronization efficiency.
\item[Read-to-Write Ratio:]  A data structure that is
	rarely updated may often be replicated rather than partitioned,
	and furthermore may be protected with asymmetric
	synchronization primitives that reduce readers' synchronization
	overhead at the expense of that of writers, thereby
	reducing overall synchronization overhead.
	Corresponding optimizations are possible for frequently
	updated data structures, as discussed in
	Chapter~\ref{chp:Counting}.
\item[Complexity:]  A parallel program is more complex than
	an equivalent sequential program because the parallel program
	has a much larger state space than does the sequential program,
	although large state spaces having regular structures can in
	some cases be easily understood.
	A parallel programmer must
	consider synchronization primitives, messaging, locking design,
	critical-section identification,
	and deadlock in the context of this larger state space.

	This greater complexity often translates
	to higher development and maintenance costs.
	Therefore, budgetary constraints can
	limit the number and types of modifications made to
	an existing program, since a given degree of speedup is
	worth only so much time and trouble.
	Worse yet, added complexity can actually \emph{reduce}
	performance and scalability.

	Therefore, beyond a certain point,
	there may be potential sequential optimizations
	that are cheaper and more effective than parallelization.
	As noted in
	Section~\ref{sec:intro:Performance},
	parallelization is but one performance optimization of
	many, and is furthermore an optimization that applies
	most readily to CPU-based bottlenecks.
\end{description}
These criteria will act together to enforce a maximum speedup.
The first three criteria are deeply interrelated, so
the remainder of this section analyzes these
interrelationships.\footnote{
	A real-world parallel system will be subject to many additional
	design criteria, such as data-structure layout,
	memory size, memory-hierarchy latencies, bandwidth limitations,
	and I/O issues.}

Note that these criteria may also appear as part of the requirements
specification.
For example, speedup may act as a relative desideratum
(``the faster, the better'')
or as an absolute requirement of the workload (``the system
must support at least 1,000,000 web hits per second'').
Classic design pattern languages describe relative desiderata as forces
and absolute requirements as context.

An understanding of the relationships between these design criteria can
be very helpful when identifying appropriate design tradeoffs for a
parallel program.
\begin{enumerate}
\item	The less time a program spends in exclusive-lock critical sections,
	the greater the potential speedup.
	This is a consequence of \IXr{Amdahl's Law}~\cite{GeneAmdahl1967AmdahlsLaw}
	because only one CPU may execute within a given
	exclusive-lock critical section at a given time.

	More specifically, for unbounded linear scalability, the fraction
	of time that the program spends in a given exclusive critical
	section must decrease as the number of CPUs increases.
	For example, a program will not scale to 10~CPUs
	unless it spends much less than one tenth of its time in the
	most-restrictive exclusive-lock critical section.
\item	Contention effects consume the excess CPU and/or
	wallclock time when the actual speedup is less than
	the number of available CPUs.
	The larger the gap between the number of CPUs
	and the actual speedup, the less efficiently the
	CPUs will be used.
	Similarly, the greater the desired efficiency, the smaller
	the achievable speedup.
\item	If the available synchronization primitives have
	high overhead compared to the critical sections
	that they guard, the best way to improve speedup
	is to reduce the number of times that the primitives
	are invoked.
	This can be accomplished by batching critical sections,
	using data ownership (see \cref{chp:Data Ownership}),
	using asymmetric primitives
	(see Section~\ref{chp:Deferred Processing}),
	or by using a coarse-grained design such as \IXh{code}{locking}.
\item	If the critical sections have high overhead compared
	to the primitives guarding them, the best way
	to improve speedup is to increase parallelism
	by moving to reader/writer locking, \IXh{data}{locking}, asymmetric,
	or data ownership.
\item	If the critical sections have high overhead compared
	to the primitives guarding them and the data structure
	being guarded is read much more often than modified,
	the best way to increase parallelism is to move
	to reader/writer locking or asymmetric primitives.
\item	Many changes that improve SMP performance, for example,
	reducing lock contention, also improve real-time
	latencies~\cite{PaulMcKenney2005h}.
\end{enumerate}

\QuickQuiz{
	Don't all these problems with critical sections mean that
	we should just always use
	non-blocking synchronization~\cite{MauriceHerlihy90a},
	which don't have critical sections?
}\QuickQuizAnswer{
	Although non-blocking synchronization can be very useful
	in some situations, it is no panacea, as discussed in
	\cref{sec:advsync:Non-Blocking Synchronization}.
	Also, non-blocking synchronization really does have
	critical sections, as noted by Josh Triplett.
	For example, in a non-blocking algorithm based on
	compare-and-swap operations, the code starting at the
	initial load and continuing to the compare-and-swap
	is analogous to a lock-based critical section.
}\QuickQuizEnd

It is worth reiterating that contention has many guises, including
lock contention, memory contention, cache overflow, thermal throttling,
and much else besides.
This chapter looks primarily at lock and memory contention.
